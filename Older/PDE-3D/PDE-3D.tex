% Typeset with XeTeX
% Allows use of system fonts rather than just LaTeX's ones
% NOTE - if you use TeXShop and Bibdesk (Mac), can complete citations
%  - open your .bib file, type~\citep{xx... and then F5 or Option-Escape
\documentclass[11pt]{article}
\usepackage[margin=1in, letterpaper]{geometry} % set page layout
%\geometry{letterpaper}  % or a4paper
\usepackage[xetex]{graphicx} % allows us to manipulate graphics.
% Replace option [] with pdftex if you don't use Xe(La)TeX
\usepackage{color}
\usepackage{indentfirst}
\usepackage{hyphenat}
\usepackage{epstopdf} % automatic conversion of eps to pdf 
\usepackage{amsmath, amssymb} % Better maths support & more symbols
\usepackage{textcomp} % provide lots of new symbols - see textcomp.pdf
% line spacing: \doublespacing, \onehalfspacing, \singlespacing
\usepackage{setspace}

\singlespacing
\usepackage{pgfplotstable}
% allows text flowing around figs
% use \begin{wrapfigure}{x}{width} where x = r(ight) or l(eft)
\usepackage{wrapfig}
\usepackage[parfill]{parskip} % don't indent new paragraphs
\usepackage{flafter}  % Don't place figs & tables before their definition 
\usepackage{verbatim} % allows \begin and \end{comment} regions
\usepackage{booktabs} % makes tables look good
\usepackage{bm}  % Define \bm{} to use bold math fonts
% linenumbers in L margin, start & end with \linenumbers \nolinenumbers,
\usepackage{lineno} % use option [modulo] for steps of 5
\usepackage[auth-sc]{authblk} % authors & institutions - see authblk.pdf
%\renewcommand\Authands{ and } % separates the last 2 authors in the list
% control how captions look; here, use small font and indent both margins by 20pt
\usepackage[small]{caption} 
\setlength{\captionmargin}{20pt}

%: FONT
% If you don't want to use system fonts, replace from here to 'Citation style' with \usepackage{Palatino} or similar
%: ************ FANCY FONTS START HERE
\usepackage[no-math]{fontspec} % 'no-math' = keep computer modern for math fonts
\usepackage{xunicode} % needed by XeTeX for handling all the system fonts nicely
\usepackage[no-sscript]{xltxtra} 
\setmonofont[Scale=0.8]{PT Serif} % typeface for \tt commands
\setsansfont[BoldFont={PT Serif Bold}, ItalicFont={PT Serif Italic}]{PT Serif} 
\defaultfontfeatures{Mapping=tex-text}
\setmainfont{Minion Pro}
%\setmainfont{Source Sans Pro}

%: ************ FANCY FONTS END HERE

%:CITATION STYLE
% natbib package: square,curly, angle(brackets)
% colon (default), comma (to separate multiple citations)
% authoryear (default),numbers (citations style)
% super (for superscripted numerical citations, as in Nature)
% sort (orders multiple cites into order of appearance in ref list, or year if authoryear)
% sort&compress: as sort, + multiple citations compressed (as 3-6, 15)
\usepackage[numbers,comma,sort&compress]{natbib}

%:SHORTCUT COMMANDS
% Maths
\newcommand{\ddt}[1]{\ensuremath{\frac{{\rm d}#1}{{\rm d}t}}}  % d/dt
\newcommand{\dd}[2]{\ensuremath{\frac{{\rm d}#1}{{\rm d}#2}}} % dy by dx  - \dd{y}{x}
\newcommand{\ddsq}[2]{\ensuremath{\frac{{\rm d}^2#1}{{\rm d}#2^2}}} % second deriv
\newcommand{\pp}[2]{\ensuremath{\frac{\partial #1}{\partial #2}}} % partial \pp{y}{x}
\newcommand{\ppsq}[2]{\ensuremath{\frac{\partial^2 #1}{\partial {#2}^2}}}
\newcommand{\superscript}[1]{\ensuremath{^{\textrm{#1}}}} %normal (non-math) font for super/subscripts in text
\newcommand{\subscript}[1]{\ensuremath{_{\textrm{#1}}}}
\newcommand{\positive}{\ensuremath{^+}}
\newcommand{\negative}{\ensuremath{^-}}
% Editing
\newcommand{\red}[1]{{\color{red}{#1}}}
\newcommand{\redtext}[1]{{\color{red}{#1}}}
\newcommand{\blue}[1]{{\color{blue}{#1}}}
\newcommand{\bluetext}[1]{{\color{blue}{#1}}}
\newcommand{\scinot}[2]{\ensuremath{#1 \times 10^{#2}}}
% Standard stuff
\newcommand{\ie}{\textit{i.e.}}
\newcommand{\etal}{\textit{et al.}}
\newcommand{\khi}{Ki67$^\text{hi}$}
\newcommand{\klo}{Ki67$^\text{lo}$}


% \begin{graybox} text \end{graybox} for text with a background colour
\definecolor{MyGray}{rgb}{0.96,0.97,0.98}
\definecolor{MyGray}{rgb}{0.96,0.90,0.98}
\makeatletter\newenvironment{graybox}{%
	\begin{lrbox}
		{\@tempboxa}\begin{minipage}[r]{0.98\columnwidth}}{\end{minipage}\end{lrbox}%
	\colorbox{MyGray}{\usebox{\@tempboxa}}
}\makeatother


%%%%%%%%%%%%%%%%%%%%%%%

\title{3 Dimensional PDE to track cell age and Ki67 expression}
\author{Sanket Rane}
\date{\today}

\begin{document} 
	\maketitle

In order to track the dynamics of cells of age `a' expressing the nuclear protein Ki67 with an intensity `k' through time `t', we use the following PDE,
\begin{equation}
	\frac{\partial u}{\partial t} +\frac{\partial u}{\partial a} - \beta \, k \, \frac{\partial u}{\partial k}  = - \delta(a) \, u.
	\label{PDE}
\end{equation}
- $\beta$ is constant and gives the rate of loss of Ki67 expression; \\
- $\delta$ varies with cell age  and gives the rate with which cells are lost from the population, either by death or by differentiation.

Total size of the population at time t is derived by integrating the cell density u(t, a, k) over all allowable cell ages such that $u(t, k) = \int u(t, a, k) da$. \\
The intensity of Ki67 $\in$ (0, 1) such that it is maximum (k=1) upon entering division which then decays exponentially with rate constant $\beta$.  
By setting a threshold of Ki67 intensity for Ki67$^{+}$ cells (e.g. k $\ge$ 0.5 $\rightarrow$ Ki67$^{+}$) \khi and \klo subsets can be binned and tracked over time t.

To solve this linear first-order PDE (eq. \ref{PDE}), we need to transform it into an ODE such as,
\begin{equation}
	\frac{d}{ds}u(t(s), a(s), k(s)) = F(u, t(s), a(s), k(s)),
	\label{ODE}
\end{equation}
along the (t(s), a(s), k(s)) \textbf{\textit{characteristic curve}}. 

Using the \textbf{\textit{chain rule}} we find, 
\begin{eqnarray*}
	\begin{aligned}
		\frac{d}{ds}u(t(s), a(s), k(s)) &= \frac{\partial u}{\partial t} \frac{dt}{ds} +\frac{\partial u}{\partial a} \frac{da}{ds} + \frac{\partial u}{\partial k} \frac{dk}{ds}. \\
		\\
		\text{If we set } \frac{dt}{ds} = 1, \frac{da}{ds} &= 1 \text{ and } \frac{dk}{ds} = - \beta \, k \text{, we get} \\
		\\
		\frac{du}{ds} &=\frac{\partial u}{\partial t} +\frac{\partial u}{\partial a} - \beta \, k \, \frac{\partial u}{\partial k}, \\
	\end{aligned}
\end{eqnarray*}
which is the LHS of the PDE that we started with in eq. \ref{PDE}.
This suggests that along the characteristic curve, the solution of the ODE (eq. \ref{ODE}) is given by,
\begin{eqnarray}
	\begin{aligned}
	&\frac{du}{ds} = - \delta(a) \, u, \\
	&\text{which can be solved to give,} \\
	&u(s) = u_0 \, exp(- \int \delta(s) ds).
	\end{aligned}
	\label{SOL}
\end{eqnarray}
Therefore, if we know $u(t_0, a_0, k_0) (\ie u_0)$ we can find $u(t(s), a(s), k(s)$ using eq. \ref{SOL}, since (t(s), a(s), k(s)) and (t(0), a(0), k(0)) both lie on the same characteristic curve.
The general solution of PDE in eq. \ref{PDE} thus can be determined by solving the system of characteristic ODEs.

\section*{Tracking cells coming from source}
We assume that cells of age 0 constantly enter the target population from the source compartment which sets one of the boundary conditions for the PDE in eq. \ref{PDE}.
\begin{equation*}
	u_{s}(t, a=0, k) = \phi(t, k)
\end{equation*}
where, $\phi$ is the function that gives the source influx.
The system of characteristic ODEs for this population:
\begin{eqnarray*}
	\begin{aligned}
	&\frac{da}{ds} = 1 \rightarrow a = s + a_0 \quad \text{since,  } a_0 = 0 \rightarrow a=s\\
	&\frac{dt}{ds} = 1 \rightarrow t = s + t_0 \quad \rightarrow t=a+t_0\\
	&\frac{dk}{ds} = -\beta \, k \rightarrow k = k_0 \, e^{-\beta \, s} \quad \rightarrow k = k_0 \, e^{-\beta \, a} \\
	\end{aligned}
\end{eqnarray*}
Therefore, $\phi(t_0, k_0) = \phi(t-a, k \, e^{\beta \, a})$.

Following form eq. \ref{SOL},
\begin{eqnarray}
	u_{s}(t, a, k) = \phi\bigg(t-a, k \, e^{\beta \, a}\bigg) \, exp\bigg(- \int_0^a \delta(\tau) d\tau\bigg). 
\end{eqnarray}

\section*{Tracking cells that entered division}

We assume that cells enter division with rate $\rho(a)$ depending on their age and will acquire the maximum Ki67 intensity \ie k=1, forming the second boundary condition for our system.
\begin{equation*}
	u_{d}(t, a, k=1) = 2 \, \rho(a) \, u(t, a, k)
\end{equation*}
These cells can be tracked using the same system of characteristic ODEs:
\begin{eqnarray*}
	\begin{aligned}
		&\frac{dk}{ds} = -\beta \, k \rightarrow k = k_0 \, e^{-\beta \, s} \quad \text{since,  } k_0 = 1 \rightarrow \frac{log(k)}{-\beta} =  s \\
		&\frac{da}{ds} = 1 \rightarrow a = s + a_0 \quad \rightarrow a = \frac{log(k)}{-\beta} + a_0\\
		&\frac{dt}{ds} = 1 \rightarrow t = s + t_0 \quad \rightarrow t = \frac{log(k)}{-\beta} + t_0\\
	\end{aligned}
\end{eqnarray*}
Therefore, $u_{d}(t_0, a_0, k_0) = 2 \, \rho(a) \, u(t + \frac{log(k)}{\beta}, a + \frac{log(k)}{\beta}, k).$

Following form eq. \ref{SOL},
\begin{eqnarray}
u_{d}(t, a, k) = 2 \, \rho(a) \, u(t + \frac{log(k)}{\beta}, a + \frac{log(k)}{\beta}, k) \, exp\bigg(- \int_0^a \delta(\tau + \frac{log(k)}{\beta}) d\tau\bigg). 
\end{eqnarray}

At any given time `t' the density of cells with age `a' and Ki67 intensity `k' is the sum of the source influx and the dividing cells.
We integrate over allowable ages to get the ki67 distribution of cells. 
The counts of \khi and \klo cells are then obtained by integrating from threshold k intensity `$\kappa$' (derived form data) to 1 and from 0 to $\kappa$, respectively. 
\begin{eqnarray}
	\begin{aligned}
	u(t, a, k) = u_s(t,a,k) + u_d(t,a,k) \\
	u(t,k) = \int_0^a u(t,a,k) \\
	u_{khi}(t) = \int_{\kappa}^1 u(t,k) \\
	u_{klo}(t) = \int_0^{\kappa} u(t,k) \\
	\end{aligned}
\end{eqnarray}









	
	
\end{document}
	